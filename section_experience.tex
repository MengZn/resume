% YAAC Another Awesome CV LaTeX Template
%
% This template has been downloaded from:
% https://github.com/darwiin/yaac-another-awesome-cv
%
% Author:
% Christophe Roger
%
% Template license:
% CC BY-SA 4.0 (https://creativecommons.org/licenses/by-sa/4.0/)
%Section: Work Experience at the top
\sectionTitle{Experience}{\faSuitcase}
%\renewcommand{\labelitemi}{$\bullet$}



\begin{experiences}
	\experience
	{Now}   {思華科技 | Golang Engineer}{現職}{Xinyi}
	{Jan 2021} {
                      \begin{itemize}
                      	\item 整合\textbf{OpenTelemetry}服務追蹤服務物行為,加快除錯流程。
                      		\begin{enumerate}                      	
                      			\item 由於系統微服務過多在除錯與追蹤上非常困難,因此我積極導入OpenTelemetry技術用以追蹤服務的調用練與相關的SQL操作。
                      			\item 由於 OpenTelemetry 當前沒有完整整合 Gorm  V1 框架,另外撰寫一個 plugins  進行整合用以追蹤  Gorm hook 事件。
                      		 \end{enumerate}
                      	\item 調教legacy系統\textbf{Redis}慢查詢事件與 cache Avalanche等異常問題。
	                      	\begin{enumerate}
        	              		\item 處理cache Avalanchea 時整合了  promethues  與alert manager 提供即時的告警讓負責的相關人員能及時收到告警通知,並且自動觸發服務熔斷與限縮請求湧入DB造成後續問題。
                      		\end{enumerate}
                      	\end{itemize}
                     }
                 {Golang, OpenTelemetry, Redis}
	\emptySeparator
	

	\experience
	{Now}   {思華科技 | Golang Engineer}{現職}{Xinyi}
	{Jan 2021} {
		\begin{itemize}
			\item 導入\textbf{Skaffold}與\textbf{Kind},加快整體開發與測試速度。
			\begin{enumerate}                      	
				\item 要開發新的功能到legacy系統中是一個十分繁瑣的事情,需要模擬產品線上的環境又有可能弄髒自己的開發環境,導入docker也需要等待image build time,讓整體開發速度下降。
				\item 因此我在導入Kind與Skaffold到每個開發者的開發流程中,修改程式碼的時候就會觸發image building ,最後部署到本地的kind  kubernetes環境中,整體動作一氣呵成以便後續功能測試。
			\end{enumerate}
			\item Skaffold扔有不足靠\textbf{Telepresence}來幫忙,接入remote kubernetes環境。
			\begin{enumerate}
				\item 開發者雖然能夠透過 Skaffold 建置與發佈image到container registry ,並且自動化部屬到本地的kubernetes環境,但這僅能模擬遠端環境無法將遠端發生的事件導流到本地來。
				\item 因應上述問題,我在Kubernetes環境部署了\textbf{Telepresence},並且在開發者環境導入了\textbf{Telepresence}來幫忙導引遠端流量到本地,利開發者可以將遠端服務流量導流到本地進行部分功能測試與hot fix修正。
			\end{enumerate}
			\item <POC> 研究基於AWS step function與\textbf{Argo Workflow}與\textbf{Knative},打造免錢的step function。
			\begin{enumerate}                      	
				\item 由於Aws step function使用上計價非常昂貴,目前開源的專案有提供類似step function的功能但無法架設在kubernetes並且無法以流量或其他策略策略部署與執行containerized step function。
				\item 因此為了應對此項需求,我研究了\textbf{Argo Workflow}一種在kubernetes上執行以DAG策略執行並回傳結果的開源軟體與\textbf{Knative}一種能以流量啟動服務的開源軟體,嘗試將兩種結合打造免錢的step function。
			\end{enumerate}
		\end{itemize}
	}
	{Kubernetes, Skaffold, Telepresence, Golang, Argo Workflow, Knative}
	\emptySeparator
  \experience
    {Jan 2021}   {遠傳電信 | Software Engineer}{1 Year}{Neihu}
    {Jan 2020} {
                      \begin{itemize}
                        \item 透過\textbf{Golang}開發\textbf{Kubernetes Operator}部署及管理內部有狀態的服務 (POC)。
                        \begin{enumerate}
                        	\item 設計admission controller 驗證使用者操作行為。
                            \item 整合Custom Resources Definition用以操作企業內部有狀態服務管理的自定義資源。
                        \end{enumerate}
                        \item 導入 \textbf{Infra as Code} 之工具\textbf{Terraform} ,以取代傳統在Azure Web console 操作並部署及維護\textbf{OpenShift}。           
                        \item 整合\textbf{Gitlab CI}取代原IT手動部署與測試行為,以加快產品的迭代速度及上線之速度。                
                        \item 協助產品開發團隊使用與導入\textbf{Kafka}、\textbf{GRPC}技術
                        \begin{enumerate}
                        	\item 分析 Consumer 吞吐量問題並提供相應的解決方案。
                            \item 搭建Kafka監控畫面以利開發團隊追蹤Brocker相關性能。
                            \item 解決Kubernetes導入GRPC負載平衡問題。
                        \end{enumerate}
                        \item 搭建混合雲(Azure+On-premise)部署策略與工具開發。
                        \item 整合\textbf{Prometheus}與Grafana同時制定告警規則,將服務異常行為與狀態推播至相關負責人之通訊軟體。
                        \item 分析現行OpenShift環境導入\textbf{Service Mesh} (Istio,Linkerd)相關問題效益以及影響範圍。
                        
                      \end{itemize}
                    }
                    {Golang, Kubernetes, Gitlab CI, OpenShift, Terraform, Kafka, Grpc}
  \emptySeparator
  \experience
    {November 2019} {鈦坦科技 | Backend Developer}{4 Months,兵役前}{Nangang}
    {August 2019}    {
                      \begin{itemize}
                        \item 透過\textbf{Golang}撰寫後台禮券發放邏輯以及串接第三方金流。
                        \item 透過\textbf{Golang}撰寫自動化測試腳本用以測試後台\textbf{WebSocket}相關邏輯。
                        \item 將傳統IT人員把執行檔手動部署的方式到Server的方式透過\textbf{Container}虛擬化技術封裝,用以加速版本迭代與更新。                	
                      \end{itemize}
                    }
                    {Golang, Websocket, Container, Auto Test}
  \emptySeparator
  \experience
    {August  2019}     { NCTUxNUTC| Cooperation plan}{跨校合作計畫}{Taichung}
    {December 2017}    {
                      \begin{itemize}
                        \item ETSI MANO 設計與開發
                        \begin{enumerate}
                        	\item 分析 Open Network Foundation開源專案 CORD 如何從OpenStack環境移植到Kubernetes平台。
                        	\item 拆解CORD \textbf{Makefile}與\textbf{Ansible}安裝過程,並將元件Container化,撰寫Kubernetes Helm腳本安裝CORD基礎元件。
                            \item 透過\textbf{Golang}撰寫Kubernetes Operator讓使用者自定義CORD相關元件以及管理服務的生命週期。
                            \item 研究在5G Core Network場景中,電信業者自建機房如何透過CORD降低營運成本,與整合Kubernetes不足之特性。
                             \item 使用\textbf{Golang}開發能夠介接OpenVswitch同時管理IP的Kubernetes CNI  Plugins。
                        \end{enumerate}
                      \end{itemize}
                    }
                    {Golang,Kubernetes, OpenStack}
\emptySeparator
\experience
    {August  2019}     { NCTUxNUTC| Cooperation plan}{跨校合作計畫}{Taichung}
    {December 2018}    {
                      \begin{itemize}
                        \item 3GPP Network Management 設計與開
                        \begin{enumerate}
                        	\item 分析國際電信3GPP核心網路管理平台所需文件
                            \item 統整符合國際標準虛擬網路管理API相關文件
                            \item 規劃系統平台所需元件以及定義相關開發流程
                        \end{enumerate}
                      \end{itemize}
                    }
                    {ETSI,3GPP,5G-Core-Network}
\emptySeparator
  \experience
    {October 2017}     { inwinSTACK 迎棧科技| Intern}{暑期實習}
    {Banqiao}
    {June 2017}    {
                      \begin{itemize}
                        \item Kubernetes Integration OpenStack with kuryr project 整合
                        \begin{enumerate}
                        	\item 分析 OpenStack虛擬網路與Kubernetes虛擬網路異同處。
                            \item 透過Open Vswitch整合OpenStack與Kubernetes網路將兩個叢集透過vlan與vxlan接通。
                            \item 透過Open Flow定義規則並使用Open Vswitch達到Kubernetes 與OpenStack網路路由與轉發。
                        \end{enumerate}
                        \item 分析Kubernetes OpenContrail Container Network Interface
                        \begin{enumerate}
                        	\item 分析OpenContrail插件運行方式。
                            \item 搭建OpenContrail CNI透過商用網路設備介接Kubernetes網路
                        \end{enumerate}
                      \end{itemize}
                    }
                    {Kubernetes, OpenStack, Open Vswitch, SDN}
  \emptySeparator
\end{experiences}
